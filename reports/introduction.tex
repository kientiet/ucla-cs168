%%%%%%%%%%%%%%%%%%%%%%%%%%%%%%%%%%%%%%%%%%%%%%%%%%%%%%%%%%%%%%%%%%%%%%%%%%%%%%%%
%2345678901234567890123456789012345678901234567890123456789012345678901234567890
%        1         2         3         4         5         6         7         8

% TEMPLATE for UCLA CS168

\documentclass[letterpaper, 11 pt, journal]{ieeeconf} 

\overrideIEEEmargins
% See the \addtolength command later in the file to balance the column lengths
% on the last page of the document

%\usepackage{graphics} % for pdf, bitmapped graphics files
%\usepackage{epsfig} % for postscript graphics files
%\usepackage{mathptmx} % assumes new font selection scheme installed
%\usepackage{times} % assumes new font selection scheme installed
%\usepackage{amsmath} % assumes amsmath package installed
%\usepackage{amssymb}  % assumes amsmath package installed

\usepackage{authblk}

\title{\LARGE \bf AI-Powered Quantitative Digital Pathology for Personalized Cancer Immunotherapy}

\author[1]{Kien H. Tiet *\thanks{* equal contribution}}
\author[1]{SukJin Jang *}
\author[1]{Tanishq Bhatia *}
\author[1]{Madhuri Suthar}
\affil[1]{University of California, Los Angeles}

\begin{document}
	
	\maketitle
	\thispagestyle{empty}
	\pagestyle{empty}

	\section{Introduction}

	The common way of testing whether a treatment would work on a cancer cell is to establish two groups of voluntary patients where one group 
	will be given the treatment while the other group is not. Then, the researchers will observe the effect of the treatment, and conducting 
	statistical tests in order to decide whether the medicine will be safe for further usage [1]. However, when the new patients request for
	the treatment, it takes the doctors a long time to look at the cancer cells to decide if the new patients' cancer cells will be suitable for 
	that particular medical treatment [2].

	With the current development of deep learning, especially computer vision, many current state-of-the-art computer vision models have been applied to 
	detect the cancer cells [2], segmenting the cancer cells or tumors [3], [4] etc. Besides these common tasks, these models are usually used as a guide 
	for another model to classify whether a particular patient's cancer cells will be benefited from the treatment such as [2] in which they used 
	another resnet18 [5] to decide the benefit of the treatment after dececting the cancer cells with different resnet18.
		
	In this project, our task is to decide whether the patient should get the treatment based on their current distributed cancer cells. To achieve this
	task, we will extend the model of [2] by applying attention-based mechanism into the models. Although the traditional convolution neural network (CNN)
	has already given the good result on this task, its drawback is the lack of the local context among cells in the picture. First of all, 
	the network will corrupt the input image into smaller features through convolution filters, and produce the only one common context vector for the whole 
	image after going through the stack of layers. This is a drawback because the network will only look at small region on the image without considering 
	the interaction between its neighbors. In the other words, the network only filters particular features about the cells. As we can see, this will not 
	be the case in medical image because the cancer cells usually interact with each other [1], [4], [6]. 
	As a result, this can lead to missing the overall picture of the cell (due to lack of context). 
	
	As demonstrated in [4], [6], [7] . Adding the attention blocks, it will yield better result because the network can give more 
	attention to the more important features and the general interaction among blocks of filters. 
	Therefore, our hypothesis with the extension of the model in [2] is that by adding attention blocks, the accuracy of the model will be improved by large margin and 
	it can be easier to interpret the result of the prediction.

	Since we follow and extend the model of [2], we will also use their dataset in order to compare the performance of our models. The dataset of [2] is orginally 
	published by GDC [12], and it records historical smaples of human coloractal cancer. Each image includes: coloractal cancer and healthy tissue [11].
	
	In addition, we also observe that with the traditional CNN (including CNN with attention blocks), the network does not strongly learn the real 
	interaction between a immuse cell and other cancer cells due to the covolve filter blocks and the lack of well-defined the relationship between 
	those two cells.  In the recent years, the graph-based CNN has started to gain better achievement compared to the traditional CNN due to the understanding of topology of 
	the dataset such as [8], [9], [10]. Therefore, we will apply this approach to solve the task we mentioned above by (1) converting the raw image to the graph 
	and (2) applying the graph-based CNN on the constructed graph by following the practice of [8].

	
	\cleardoublepage
	
	

	\begin{thebibliography}{12}
		
		\bibitem{c1} Neoadjuvant paper

		\bibitem{c2} Deep learning can predict

		\bibitem{c3} VNAS

		\bibitem{c4} Multi-sclae self-guided attention

		\bibitem{c5} Resnet18

		\bibitem{c6} Attention Gated Networks

		\bibitem{c7} Attention to Scale:

		\bibitem{c8} Semi-Supervised Classification

		\bibitem{c9} Convolutional Neural Networks

		\bibitem{c10} Neural Relational Inference

		\bibitem{c11} Website: Histological images for tumor detection

		\bibitem{c12} Website: https://portal.gdc.cancer.gov/
		
	\end{thebibliography}
	
\end{document}
