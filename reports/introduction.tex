%%%%%%%%%%%%%%%%%%%%%%%%%%%%%%%%%%%%%%%%%%%%%%%%%%%%%%%%%%%%%%%%%%%%%%%%%%%%%%%%
%2345678901234567890123456789012345678901234567890123456789012345678901234567890
%        1         2         3         4         5         6         7         8

% TEMPLATE for UCLA CS168

\documentclass[letterpaper, 10 pt, journal]{ieeeconf} 

\overrideIEEEmargins
% See the \addtolength command later in the file to balance the column lengths
% on the last page of the document

%\usepackage{graphics} % for pdf, bitmapped graphics files
%\usepackage{epsfig} % for postscript graphics files
%\usepackage{mathptmx} % assumes new font selection scheme installed
%\usepackage{times} % assumes new font selection scheme installed
%\usepackage{amsmath} % assumes amsmath package installed
%\usepackage{amssymb}  % assumes amsmath package installed

\usepackage{authblk}

\title{\LARGE \bf AI-Powered Quantitative Digital Pathology for Personalized Cancer Immunotherapy}

\author[1]{Kien H. Tiet *\thanks{* equal contribution}}
\author[1]{SukJin Jang *}
\author[1]{Tanishq Bhatia *}
\author[1]{Madhuri Suthar}
\affil[1]{University of California, Los Angeles}

\begin{document}
	
	\maketitle
	\thispagestyle{empty}
	\pagestyle{empty}

	\section{Introduction}

	The common way of testing whether a treatment for cancer is effective is to establish two groups of voluntary patients where one group is given the treatment while the other group is not. 
	Then, the researchers will observe the effect of the treatment, and conduct statistical tests in order to decide whether the treatment is safe for further usage [1]. However, when the new 
	patients request for the treatment, it takes the doctors a long time to look at the cancer cells to decide if the new patients' cancer cells will be suitable for that particular medical 
	treatment [2]. Besides, due to the high cost and the long process, many patients do not get the treatment on time. Thus, cancerous cells impair the immune system and prevent the body 
	from functioning properly [12].	
	
	With the current development of deep learning, especially computer vision, many current state-of-the-art computer vision models 
	have been applied to detect the cancer cells [2], segmenting the cancer cells or tumors [3], [4] etc. Besides these common tasks, 
	these models are usually used as a guide for another model to classify whether a particular patient's cancer cells will be benefited from the 
	treatment such as [2] in which they used another resnet18 [5] to decide the benefit of the treatment after detecting the cancer cells 
	with different resnet18.		
	
	In this project, our task is to decide whether the patient should get the treatment based on their current distributed cancer cells. 
	To achieve this task, we will extend the model of [2] by applying attention-based mechanisms into the models. Although the traditional 
	convolution neural network (CNN) has already given a good result on this task, its drawback is the lack of the local context among cells in 
	the picture. First of all, the network will corrupt the input image into smaller features through convolution filters, and produce the only one 
	common context vector for the whole image after going through the stack of layers. This is a drawback because the network will only look at a 
	small region on the image without considering the interaction between its neighbors. In the other words, the network only filters particular features 
	about the cells. As we can see, this will not be the case in medical image because the cancer cells usually interact with each other [1], [4], [6]. 
	As a result, this can lead to missing the overall picture of the cell (due to lack of context).	
	
	As demonstrated in [4], [6], [7], adding the attention blocks, it will yield better results because the network can give more attention to the 
	more important features and the general interaction among blocks of filters. Although most of these works were used for segmentation purpose instead 
	of giving the decision whether the patient should get the treatment or not,.  we will extend of the model in [2] by adding attention blocks (for both channels 
	and features). We also hypothesize that the accuracy of the model will be improved by large margin and it can be easier to interpret the result of the 
	prediction. The reason is by adding attention blocks, we can give more relevant features and channels with higher weight, and we can also track the heatmap or 
	activiation map of these attention blocks in order to give us the interpretation of the inference results.
	
	Since we follow and extend the model of [2], we will also use their dataset in order to compare the performance of our models. The dataset of [2] 
	is originally published by National Cancer Institutes (https://portal.gdc.cancer.gov/), and it records historical samples of human colorectal cancer. 
	Each image includes: colorectal cancer and healthy tissue
	\footnote{http://www.andrewjanowczyk.com/download-tcga-digital-pathology-images-ffpe/}.
	
	In addition, we also observe that with the traditional CNN (including CNN with attention blocks), the network does not strongly learn the real 
	interaction between an immune cell and other cancer cells due to the convolve filter blocks and the lack of well-defined relationship between those 
	two cells.  In recent years, the graph-based CNN has started to gain better achievement compared to the traditional CNN due to the understanding of 
	topology of the dataset such as [8], [9], [10]. Therefore, we will apply this approach to solve the task we mentioned above by (1) converting the raw 
	image to the graph and (2) applying the graph-based CNN on the constructed graph by following the practice of [8].
	
	\cleardoublepage
	
	

	\begin{thebibliography}{12}
		
		\bibitem{c1} Cloughesy, T.F., Mochizuki, A.Y., Orpilla, J.R. et al. Neoadjuvant anti-PD-1 immunotherapy promotes a survival benefit with intratumoral and systemic immune responses in recurrent glioblastoma. Nat Med 25, 477–486 (2019). https://doi.org/10.1038/s41591-018-0337-7

		\bibitem{c2} Kather, J.N., Pearson, A.T., Halama, N. et al. Deep learning can predict microsatellite instability directly from histology in gastrointestinal cancer. Nat Med 25, 1054–1056 (2019). https://doi.org/10.1038/s41591-019-0462-y

		\bibitem{c3} Z. Zhu, C. Liu, D. Yang, A. Yuille and D. Xu, "V-NAS: Neural Architecture Search for Volumetric Medical Image Segmentation," 2019 International Conference on 3D Vision (3DV), Québec City, QC, Canada, 2019, pp. 240-248.

		\bibitem{c4} A. Sinha and J. Dolz, "Multi-scale self-guided attention for medical image segmentation," in IEEE Journal of Biomedical and Health Informatics.

		\bibitem{c5} He, Kaiming et al. “Deep Residual Learning for Image Recognition.” 2016 IEEE Conference on Computer Vision and Pattern Recognition (CVPR) (2016): 770-778.

		\bibitem{c6} Schlemper, Jo & Oktay, Ozan & Schaap, Michiel & Heinrich, Mattias & Kainz, Bernhard & Glocker, Ben & Rueckert, Daniel. (2019). Attention Gated Networks: Learning to Leverage Salient Regions in Medical Images. Medical Image Analysis. 53. 10.1016/j.media.2019.01.012. 

		\bibitem{c7} L.-C. Chen, Y. Yang, J. Wang, W. Xu, and A. L. Yuille, “Attention to scale: Scale-aware semantic image segmentation,” in Proc. IEEE Conf. Comput. Vis. Pattern Recog., 2016

		\bibitem{c8} Kipf, Thomas and Max Welling. “Semi-Supervised Classification with Graph Convolutional Networks.” ArXiv abs/1609.02907 (2017).

		\bibitem{c9} Michael Defferrard, Xavier Bresson, and Pierre Vandergheynst. Convolutional neural networks on ¨ graphs with fast localized spectral filtering. In Advances in neural information processing systems (NIPS), 2016.

		\bibitem{c10} T. Kipf, E. Fetaya, K.-C. Wang, M. Welling, and R. Zemel, “Neural relational inference for interacting systems,” arXiv preprint arXiv:1802.04687, 2018.

		\bibitem{c11} “Cancer Stat Facts: Common Cancer Sites.” National Cancer Institute, seer.cancer.gov/statfacts/html/common.html.

		\bibitem{c12} Ranchod, Yamini. “What to Know about Cancer.” Medical News Today, 6 Jan. 2020, www.medicalnewstoday.com/articles/323648.
		
		
	\end{thebibliography}
	
\end{document}
